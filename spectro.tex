%! TeX program = lualatex
%---------------------------ALLGEMEINE IMPORTS-------------------------------------
\documentclass[12pt,english,ngerman]{scrartcl}
\input{./protokoll_template/template.latex/input/shared_preamble.tex}

% Kopfzeile
\ihead{WS22\\
	16.12.2022} \chead{\textsc{Stark} Matthias - 12004907 \\
	\textsc{Philipp} Maximilian - 11839611}
\ohead{FLAB 1 \\
	Spektrograph}
% Fußzeile


\begin{document}

\section{Aufgabenstellung\label{Auf}}

\begin{itemize}
	\item Adjustierung des Prismenspektrographen, der Linsen und der Prismen im Spektrographen, um ein Spektrum für alle 
	Lichtquellen abzubilden.
	\item Aufnahme des Spektrums von den Verschiedenen Lichtquellen und der Probe.
	\item Aufnahme der Spektren der gleichen Lichtquellen mit dem Gitterspektrographen.
	\item Erzeugung der Dispersionskurve des Prismenspektrographen.
	\item Bestimmung des Auflösevermögend der beiden Spektrographen.
	\item Bestimmung der Wellenlänge der Jod-Absorbtionsbandkanten.
	\item Bestimmung der Disotiationsenergie des Jodmoleküls.
\end{itemize}

\section{Grundlagen}\label{Grund}


Molekulare Spektroskopie

Wenn elektromagnetische Strahlung mit einem Molekül in Wechselwirkung tritt, kann sie absorbiert, emittiert oder gestreut werden, je nach der Energie der Strahlung und den Energieniveaus des Moleküls. Die Energie der Strahlung wird durch ihre Wellenlänge oder Frequenz bestimmt, und die Energieniveaus des Moleküls werden durch seine elektronische Struktur, seine Vibrations- und Rotationsmoden bestimmt.

Die Absorption oder Emission elektromagnetischer Strahlung durch ein Molekül kann als Änderung der Intensität der Strahlung beim Durchgang durch die Probe beobachtet werden. Diese Änderung kann mit einem Spektrometer gemessen und zur Untersuchung der elektronischen Struktur und Bindung des Moleküls sowie seiner Schwingungs- und Rotationsenergieniveaus verwendet werden.

Prismenspektrograf
Ein Prismenspektrograf ist ein Spektrograf, der ein Prisma verwendet, um die elektromagnetische Strahlung in ihre einzelnen Wellenlängen oder Frequenzen zu zerlegen. Er besteht aus einer Lichtquelle, einem Prisma und einem Detektor und wird verwendet, um das elektromagnetische Spektrum einer Probe grafisch darzustellen.

Die Lichtquelle eines Prismenspektrographen ist in der Regel eine Lampe oder ein Laser, mit dem die Probe beleuchtet wird. Die Probe wird in den Strahlengang gestellt, und ein Teil der elektromagnetischen Strahlung wird von der Probe absorbiert, emittiert oder gestreut.

Das Prisma ist ein transparentes optisches Element, das aus einem brechenden Material wie Glas oder Quarz besteht. Es wird in den Weg der elektromagnetischen Strahlung gestellt und dient dazu, die Strahlung in ihre einzelnen Wellenlängen oder Frequenzen zu zerlegen. Das Prisma funktioniert, indem es die Strahlung je nach Wellenlänge oder Frequenz in unterschiedlichen Winkeln bricht. Dies führt dazu, dass sich die verschiedenen Wellenlängen oder Frequenzen der Strahlung ausbreiten und ein Spektrum bilden.

Der Detektor eines Prismenspektrographen ist in der Regel eine Kamera oder ein Aufzeichnungsgerät, mit dem die Intensität der elektromagnetischen Strahlung über einen Bereich von Wellenlängen oder Frequenzen gemessen wird. Der Detektor erzeugt ein Diagramm der Intensität der Strahlung in Abhängigkeit von der Wellenlänge oder Frequenz, das als Spektrum bezeichnet wird.

Gitterspektrograf
Ein Echelle-Spektrograf ist ein Spektrograf, der ein Beugungsgitter, das so genannte Echelle-Gitter, verwendet, um die elektromagnetische Strahlung in ihre einzelnen Wellenlängen oder Frequenzen zu zerlegen. Er besteht aus einer Lichtquelle, einem Echelle-Gitter, einem Kollimator, einer Kamera oder einem Aufzeichnungsgerät und einer Brennebene.

Bei der Lichtquelle eines Echelle-Spektrographen handelt es sich in der Regel um eine Lampe oder einen Laser, mit dem die Probe beleuchtet wird. Die Probe wird in den Strahlengang gestellt, und ein Teil der elektromagnetischen Strahlung wird von der Probe absorbiert, emittiert oder gestreut.

Das Echelle-Gitter ist ein Beugungsgitter, das aus einem transparenten optischen Element wie Glas oder Quarz besteht und mit einem periodischen Muster aus parallelen Linien versehen ist. Es wird im Strahlengang der elektromagnetischen Strahlung angebracht und dient dazu, die Strahlung in ihre einzelnen Wellenlängen oder Frequenzen zu zerlegen. Das Echelle-Gitter funktioniert, indem es die Strahlung je nach ihrer Wellenlänge oder Frequenz in unterschiedlichen Winkeln beugt. Dies führt dazu, dass sich die verschiedenen Wellenlängen oder Frequenzen der Strahlung ausbreiten und ein Spektrum bilden.

Der Kollimator ist eine Linse oder ein Spiegel, der dazu dient, die elektromagnetische Strahlung zu kollimieren, d. h. zu einem parallelen Strahl zu bündeln. Der Kollimator befindet sich hinter dem Echelle-Gitter und sorgt dafür, dass die Strahlung auf die Kamera oder den Rekorder fokussiert wird.

Mit der Kamera oder dem Rekorder wird die Intensität der elektromagnetischen Strahlung über einen Bereich von Wellenlängen oder Frequenzen gemessen. Es wird ein Diagramm der Strahlungsintensität in Abhängigkeit von der Wellenlänge oder Frequenz erstellt, das als Spektrum bezeichnet wird.

Die Brennebene ist die Ebene, in der das Spektrum gebildet wird, und sie befindet sich in der Regel im Brennpunkt des Kollimators. Die Kamera oder der Rekorder wird in der Fokusebene platziert, um die Intensität der elektromagnetischen Strahlung zu messen.

\end{document}